
%\renewcommand{\figurename}{Figur}
\hyphenation{informations-teknologin}
\hyphenation{informations-säkerhet}
\hyphenation{informations-tekno-login}

\chapter*{Populärvetenskaplig sammanfattning}

Digitaliseringen av samhället fortsätter i allt snabbare takt, och informationsteknologin utvecklas kontinuerligt.
De första stegen var att digitalisera lokala samlingar av information för att göra dem enkelt sökbara på plats.
Med informationen väl digitaliserad var nästa steg istället att göra informationen tillgänglig för andra, till exempel över Internet.

Med ökad tillgänglighet krävs dock ökad informationssäkerhet; en ökad tillgänglighet leder också till en ökad risk för att informationen kan ändras, eller hamna i orätta händer.
Nästa steg inom databehandling -- som pågår i detta nu -- är den ökande användningen av IT-tjänster som hanteras av någon annan, populärt kallat \emph{molntjänster}.
Dessa tjänster har snabbt växt i populäritet, mycket beroende på ökad flexibilitet, då användare enkelt kan öka eller minska den inköpta kapaciteten snabbt efter behov.

Behandling och lagring av data på servrar enbart nåbara över Internet kräver förstås att informationssäkerhet diskuteras.
Denna avhandling berör informationssäkerhet i dessa moderna situationer.

\section*{Säker kommunikation}

Ett grundläggande krav för att lagra och hämta information från tjänster på Internet är att informationen kan överföras på ett säkert sätt.
Vad som menas med \emph{säkert} varierar från fall till fall, men ett vanligt krav är att konfidentialiteten ska skyddas, det vill säga att informationen enbart ska kunna läsas av behöriga användare.
I praktiken uppnås detta ofta med hjälp av \emph{kryptering}, ett sätt att göra information oläsbar för obehöriga som övervakar trafiken; endast personer med rätt nyckel kan läsa informationen.

Design av dessa algoritmer kräver ofta att hänsys tas till flera olika aspekter.
I de allra flesta fall är det inte enbart säkerhet som är relevant, utan även prestandan hos algoritmen.
Algoritmen behöver ha en tillräcklig säkerhetsmarginal för att kunna ge trovärdig säkerhet, men samtidigt inte ha en onödigt hög marginal eftersom det ger sämre prestanda.

Ett sätt att analysera säkerheten i en krypteringsalgoritm är att försöka konstruera en \emph{urskiljare}.
Målet med denna är att avgöra om en given mängd data är resultatet av en kryptering, eller om det bara är slumpmässig data.
För en väldesignad krypteringsalgoritm ska det inte vara möjligt att konstruera en urskiljare.
En av artiklarna i denna avhandling berör konstruktionen av sådana urskiljare, närmare bestämt en algoritm för att hitta en delmängd av insignaler till krypteringsalgoritmen som får den att uppvisa maximal icke-slumpmässighet.
Sådan icke-slumpmässighet kan sedan användas för att försöka konstruera en urskiljare, eller designa andra typer av attacker.
Den konstruerade algoritmen är en generalisering av tidigare naivt giriga algoritmer, och ger ett bättre resultat eftersom den minskar risken att fastna i lokala extrempunkter.

\section*{Tillförlitlig databehandling}

Molntjänsters ökade flexibilitet för användaren är en uppenbar fördel, samtidigt som det också medför ny problematik för användarna.
Eftersom data nu behandlas på någon annans system -- i motsats till användarens egna -- blir en rimlig följdfråga hur systemets beteende kan garanteras.

En teknik för att garantera datorers beteende är \emph{trusted computing} -- ungefär \emph{tillförlitlig databehandling} -- ofta förkortat TC.
I praktiken finns det flera olika sätt att implementera TC i datorer: två vanliga implementationer är Trusted Platform Module (TPM), och Software Guard Extensions (SGX).
TPM är en separat hårdvarukomponent, inbyggd i många moderna datorer, som kan utföra kryptografiska beräkningar och lagra information.
Komponenten är designad för att vara svår att manipulera, vilket betyder att även någon med fysisk tillgång har svårt att läsa ut känslig information.
Det gör att en TPM ofta används för att lagra känslig information så som krypteringsnycklar.

Problem uppstår dock om datorn av någon anledning behöver bytas ut, eller om komponenten av någon anledning slutar fungera.
I dessa fall behöver den känsliga informationen flyttas -- \emph{migreras} -- till en ny enhet.
I en av artiklarna i denna avhandling tittar vi hur detta kan lösas i system som kräver hög tillgänglighet.
Dessutom beskrivs hur problemet kan lösas för två olika versioner av TPM-standarden, TPM 1.2 och TPM 2.0.

En annan av artiklarna i avhandlingen beskriver hur en migrering från TPM 1.2 till TPM 2.0 kan se ut.
TPM 2.0 är den nyare standarden, men är inte bakåtkompatibel -- det finns betydande skillnader i beteende mellan de båda versionerna.
I artikeln beskrivs hur nycklar kan flyttas från TPM 1.2 till TPM 2.0 med bibehållen funktionalitet.

För att kunna leverera den flexibilitet som krävs av molntjänster används idag \emph{mjukvarudefinierade nätverk} (eng. Software-defined networking).
Dessa möjliggör för tjänsteleverantörer att dynamiskt bygga upp nätverk med hjälp av mjukvara, istället för att fysiskt behöva koppla om nätverksutrustning.
I en annan av artiklarna i denna avhandling beskrivs en lösning för att verifiera sådana mjukvarubaserade nätverkskomponenter \emph{innan} de ansluts till nätverket.
För att möjliggöra detta används trusted computing.
Detta ökar säkerheten i nätverket, eftersom obehöriga nätverkskomponenter utgör en säkerhetsrisk.

\section*{Mjukvarusårbarheter}

Vid utveckling av modern mjukvara används i stor utsträckning externa mjukvarukomponenter, ofta utvecklade av andra än utvecklaren själv.
Detta gör utvecklingsprocessen snabbare, minskar risken att upprepa samma misstag som andra, och ger tid att fokusera på utveckling av ny funktionalitet.
Användning av externa komponenter kräver dock att utvecklaren uppdaterar dessa externa komponenter när säkerhetsbrister upptäcks -- på liknande sätt som datoranvändare behöver installera säkerhetsuppdateringar för sina program.

I avhandlingen beskrivs ett hjälpmedel för utvecklare att bedöma hur relevant en mjukvarusårbarhet är för dem.
Systemet samlar information om utvecklarens preferenser och lär sig hur utvecklaren brukar agera för en viss typ av mjukvarusårbarhet, och skapar utifrån detta en profil.
Systemet kan användas som beslutsstöd för att underlätta för utvecklare att bestämma hur en sårbarhet ska hanteras.

Slutligen beskriver en av avhandlingens artiklar hur tillförlitlig databehandling kan användas för att skydda profilen i ett system som det ovan.
Utvecklarens profil kan uppfattas som känslig, och bör därför inte spridas.
Samtidigt behövs profilen för att systemet ska kunna ge bra rekommendationer.
Därför har en lösning designats så att profilen kan användas av systemet, utan att det går att koppla en given profil till en viss utvecklare.